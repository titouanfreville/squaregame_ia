\documentclass[a4paper,12pt]{report}
\usepackage[utf8]{inputenc}
\usepackage{xspace}
\usepackage[francais]{babel}

% Title Page
\title{Rapport technique \\Mini Projet : Le jeu du carré, IA joueuses}
\author{Ange Jennyfer NGUENO FOKAM - 170840\\Julien PETRIGNET 217307\\Titouan FREVILLE - 217821\\Sara EL AICHI 226328}


% Document start
\begin{document}
% Title :O
\maketitle
% contents
\tableofcontents

% R\'esum\'e
\begin{abstract}
Le projet IA joueuses pour le jeu du carré à pour but de répondre au sujet de 4AIT de l'école Supinfo, promotion 2018. L'objectif du projet est de réalisé aux moins deux intelligences artificielle capable de jouer au jeu du carré. Les intelligences propos\'ees devront \^etre capable de jouer entre elle ou contre un joueur ext\'erieur. \\
\\
A partir de cet \'enonc\'e, plusieurs solutions s'offrent \`a nous. Pour pouvoir r\'epondre le mieux aux probl\`emes, il nous faut tout d'abord analyser le jeu s\'electionner afin de d\'eterminer son type (d\'ecision, for\c{c}age, anticipation, calcul, r\'eaction, ...). Nous pourrons alors nous interroger sur les diff\'erentes IA proposable puis sur les langages utilisables afin de s\'electionner le plus intéressant pour nous.
\\
Ce rapport \`a pour but de pr\'esenter toute la d\'emarche de r\'eflexion que nous avons eu afin de prendre une d\'ecision nous permettant de r\'esoudre le probl\`eme de la fa\c{c}on la plus intéressante.
\end{abstract}

% ------------------------------------------------------------- PRESENTATION GENERALE JEU -------------------------------------------------------------  %
\part{Le jeu, pr\'esentation et compr\'ehension}

\chapter{Historique, principe de base, type}
Le jeu du carr\'e est un jeu d\'ecris en 1889 pour la premi\`ere fois par Edouard Lucas, math\'ematicien. Le principe du jeu est tr\`es simple, et nécessite peu de mat\'eriel. Le but est de cr\'eer des forme carr\'e. Pour cela, chaque joueur va a tour de r\^ole tracer un segment sur un quadrillage permettant de repr\'esenter un c\^ot\'e pour un future carr\'e. Le jouer ayant ferm\'e le plus de carr\'e remporte la partie.

\section{Type de jeu}

Le jeu des petits carr\'es est un jeu de r\'eflexion. Il fonctionne comme le jeu de dame, dans le sens ou, lorsque le jeu est ma\^itriser, l'objectif devient de forcer le jouer adverse a fermer un certain nombre de carr\'e pour pouvoir en r\'ecup\'erer plus par la suite. C'est donc un jeu de for\c{c}age et de d\'ecision. Fonctionnant sur des formes g\'eom\'etrique, il va donc se concentr\'e sur la capacit\'e \`a lire le plan repr\'esenter par la grille et \`a venir correctement enfermer l'adversaire dans une situation de fermeture perdante. Nous allons donc chercher \`a permettre \`a nos IA une bonne compr\'ehension de l'occupation de l'espace, et une bonne lecture du plan grillag\'e.

\section{Compr\'ehension}

D\'etaillons maintenant la strat\'egie principale du jeu.\\
Le jeu se base donc sur la g\'eom\'etrie et plus particuli\`erement les formes rectangulaires, et les carr\'es. La premi\`ere phase du jeu va avoir pour objectif de cr\'eer des zones de "<non droit"> tel que si un joueur place un trait dans cette zone, il donne une grande quantit\'e de points \`a son adversaire. Ensuite, chaque joueur va devoir essayer d'agrandir sa zone de non droit et en "<poss\'eder"> une depuis la qu'elle il r\'ecup\'erera plus de points que son adversaire. L'objectif est donc de cr\'eer un couloir (appeler plus g\'en\'eralement "<serpent"> en raison de sa forme) longiligne puis de forcer l'adversaire \`a jouer en bordure de ce couloir tel que d\`es qu'il ferme un carr\'e dans cette zone, le joueur r\'ecup\`ere plus de point que lui. \\
Cette strat\'egie essentiel est toutefois complexe, et n'est pas ce que les joueurs seront capable de produire en premier. Afin de proposer des IAs \'equilibr\'ees,  il nous faudra donc une IA incapable de pr\'evoir/comprendre ce Fonctionnement. \\
Nous pouvons d\'ej\`a entrevoir deux syst\`eme pour concevoir nos IA: une IA simpliste compl\'etant la grille et fermant les carr\'es d\`es que possible, et une IA plus complexe g\'erant la notion basique de serpent sans anticiper sur les coups \`a venir.
% -----------------------------------------------------------------------------------------------------------------------------------------------------  %
% ------------------------------------------------------------- PRESENTATION IA  ----------------------------------------------------------------------  %
\part{Intelligence Artificielle, let's play}
\chapter{D\'efinition des IA}

Dans la partie pr\'ec\'edente, nous avons parl\'e de plusieur IA possible, plus pr\'ecis\'ement, nous avons introduit deux IA simple. Nous allons ici continuer ce travail afin de red\'efinir les IAs et les compl\'eter.


% ---- IA 0 ------------------------------------------------------------------------
\section{IA - Base. Niveau 0}

La premi\`ere IA que nous allons d\'ecrire servira de base \`a toutes les IAs suivantes. C'est l'IA la plus simple que l'on puisse faire pour obtenir un r\'esultat interressant, et ce n'est pas pleinement une IA dans le sens ou elle va simplement automatiser un processus de traitement de donn\'ee. \\
L'unique objectif de cette IA est d'\^etre capable de jouer au jeu.

\subsection{Principe}

L'IA basique a donc pour objectif simple de pouvoir placer correctement les liasons sur la grille, et \^etre capable de prendre des points. Pour ce faire, elle va se contenter de placer les liaisons hors carr\'e (des liaisons entre deux somets sont c\^ot\'e) de fa\c{c}on al\'eatoire sur la grille. Puis elle compl\'etera les carr\'es ayant un c\^ot\'e. Une priorit\'e absolue est donn\'e pour fermer les carr\'es, ce qui im\`plique que l'IA ferme de fa\c{c}on syst\'ematique les carr\'es visible jouable \`a son tour. Elle est incapable de calculer le revenu (en point d'une de ses actions).

\subsection{Heurisitque}

Cette IA a donc besoin de peu de connaissance, et peux de processus. Ces connaissances seront par la suite pr\'esente dans toutes les IAs. \\

\begin{itemize}
 \item Capable de relier deux sommets correctement
 \item Capable de lire la grille
 \item Capable de trouver les carr\'es fermable
\end{itemize}

Ces trois \'el\'ements sont suffisant pour que cet IA puisse fonctionner.

\subsection{Algorithmes}

\subsection{Complexit\'e}

\subsubsection{Spaciale}

\subsubsection{Temporelle}

% ---- IA 1 ------------------------------------------------------------------------
\section{IA - Premi\`eres armes. Niveau 1}

D\'eveloppons maintenant une IA utilisant les structures de serpents introduite en premi\`ere partie. 

\subsection{Principe}

Cette seconde IA se base sur les m\^emes connaissances que l'IA pr\'ec\'edente. Cependant, nous allons modifier les r\`egles de placement des liaisons et ajouter la capacit\'e \`a compter les points. Ainsi, plut\^ot que de placer de fa\c{c}on al\'eatoire des c\^ot\'es, cette nouvelle IA va chercher \`a cr\'eer des serpents et des coridors. Par cons\'equents, elle va en priorit\'e placer un segment au bout d'un autre segment en cherchant a relier deux sommets sp\'ecifiques. Le choix de liaisons entre deux sommets se fait sur la quantit\'e de points r\'ecup\'erables \`a la fermeture du serpent, cette valeur devant \^etre maximis\'ee. Lorsque l'IA \`a la possibilit\'e de clore un serpent, elle va s'interroger sur la quantit\'e de points qu'elle va gagner en le fermant par rapport \`a la quantit\'e de point que l'adversaire peut r\'ecup\'erer. Si le rapport est positif, elle fermera syst\'ematiquement le serpend

\subsection{Heurisitque}

Notre seconde IA a donc besoin de nouvelle connaissance, bien qu'elle reste tr\`es simpliste.

\begin{itemize}
 \item Capable de choisir le meilleur segment d'un serpent
 \item Capable de compter les points gagnables par chaque partie dans une configuration connue
\end{itemize}

Il lui faut donc deux nouvelles connaissance pour fonctionner correctement.
\subsection{Algorithmes}

\subsection{Complexit\'e}

\subsubsection{Spaciale}

\subsubsection{Temporelle}

% ---- IA 2 ------------------------------------------------------------------------
\section{IA - La connaissance. Niveau 2}

Pour le joueur, l'IA 1 va \^etre plus complexe a vaincre par sa capacit\'e \`a choisir l'option avec meilleur gain. Cependant, elle reste extr\`emement simple \`a vaincre car tr\`es sensible au forcing. Ce qui nous am\`ene \`a une nouvelle IA. 

\subsection{Principe}

L'IA que nous allons introduire ici est l'IA la plus compl\`ete que l'on puisse faire pour un jeu. Elle est simple \`a comprendre, mais extr\`emement difficile \`a battre. Elle consiste \`a g\'en\'erer l'int\'egralit\'e des \'evolutions possibles dans une grille de la grille vide a la grille pleine. Elle connait alors toutes les configurations possibles \`a chaque instant de la partie et vas toujours choisir le placement lui donnant le plus de condition de victoire \`a chaque instant.

\subsection{Heurisitque}

Notre troisi\`eme IA a donc besoin d'une seule chose en plus de l'IA 0: la connaissance de toutes les configurations. Elle n'a plus besoin de savoir compter les points puisque elle sait qu'elle coup l'am\`ene \`a la victoire. Elle n'a plus non plus besoin de savoir construire des serpents. Bref, il lui faut juste revenir aux connaissance de bases et cr\'eer l'arbre de connaissance correspondant \`a la grille demand\'e par le joueur.

\subsection{Algorithmes}

\subsection{Complexit\'e}

\subsubsection{Spaciale}

\subsubsection{Temporelle}

% ------------------------------------------------------------- MODELISATION -------------------------------------------------------------  %
\part{Mod\'elisation}

Maintenant que nos futures IA sont d\'efinis, voyons comment mod\'eliser les donn\'ees du probl\`emes.

\chapter{Environement}

L'environement de jeu consiste en un plateau pr\'esentant une grille. C'est sur cette grille que nos viendrons dessiner nos segment. Le plateau globale est de forme rectangulaire, id\'ealement carr\'e. La grille est d\'ecouper en carr\'e, d\'ecomposer en ligne et somment. Plusieur solutions s'offrent alors \`a nous pour repr\'esenter l'environement. Nous pouvons d\'efinir un graphe contenant chaque sommet et les liasions entre eux, en utilisant une variable sur chaque noeud pour pr\'esenter la pr\'esence ou non d'une liaison, ou nous pouvons repr\'esenter chaque sommet de la grille dans une matrice, et utiliser une liste pour stocker les liaisons connue. \\
\section{Tableau}
La repr\'esentation via tableau est la repr\'esentation la plus informatique du probl\`eme, et la plus simple vis \`a vis de la repr\'esentation graphique du probl\`eme. Chaque point de la matrice repr\'esentant un sommet de la grille, il est facile de s\'electionner des sommets puis de les liers dans une liste via un couple.\\
L'avantage de cette m\'ethode est donc la facilit\'e de gestion de la partie graphique du jeu. Cependant, il est alors compliquer de savoir qu'elle c\^ot\'e sont d\'ej\`a tracer. Il est donc plus long de r\'ecup\'erer les informations quand aux corridors et serpents pr\'esent de le jeu, ainsi que de s'assurer que le coup est autoris\'e.

\section{Graphe}

\end{document}
